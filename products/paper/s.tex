PONTIFICIA UNIVERSIDAD CATÓLICA DEL PERÚ

FACULTAD DE CIENCIAS SOCIALES

La brecha digital en los estudiantes de educación básica regular de Perú
en tiempos de Covid-19

TRABAJO DE INVESTIGACIÓN PARA OBTENER EL GRADO DE BACHILLER EN CIENCIAS
SOCIALES CON MENCIÓN EN ECONOMÍA

AUTOR

Bruno Apaza, Angella Vanessa

ASESOR Sotomayor Valenzuela, Narda Lizette

Lima, diciembre de 2020 ÍNDICE DE CONTENIDOS 1. INTRODUCCIÓN 4 2.
REVISIÓN DE LITERATURA 8 2.1. SOCIEDAD DE LA INFORMACIÓN Y BRECHA
DIGITAL: BENEFICIOS Y BARRERAS 8 2.2. DEFINICIÓN DE BRECHA DIGITAL Y SUS
DIMENSIONES 9 2.3. DETERMINANTES DE ACCESO, USO DE LAS TIC POR PARTE DE
LOS ESTUDIANTES EN TIEMPOS DE CONFINAMIENTO 12 2.3.1. DETERMINANTES
SOCIOECONÓMICOS Y EDUCATIVOS 12 2.3.2. DETERMINANTES SOCIODEMOGRÁFICOS
13 2.3.3. DETERMINANTES GEOGRÁFICOS 15 2.4. BALANCE DE LA REVISIÓN DE
LITERATURA 16 3. MARCO TEÓRICO 18 3.1. MODELO TEÓRICO 18 3.2. BRECHAS:
ACCESO, USO E INTENSIDAD DE USO 19 3.3. DETERMINANTES DE ACCESO, USO Y
APROPIACIÓN DE LOS ESTUDIANTES 20 4. HECHOS ESTILIZADOS 23 4.1. DATOS 23
4.2. SITUACIÓN EN LA REGIÓN 24 4.2.1. POLÍTICAS TIC EN LA EDUCACIÓN 24
4.2.2. AVANCES Y LIMITACIONES DE LA DIGITALIZACIÓN EN EDUCACIÓN EN LA
PANDEMIA 25 4.3 SITUACIÓN PERUANA 27 5. HIPÓTESIS 31 6. LINEAMIENTO
METODOLÓGICOS 32 6.1. MUESTRA 32 6.2. METODOLOGÍA Y MODELO EMPÍRICO 33
7. CONCLUSIONES 36 8. BIBLIOGRAFÍA 37 9. ANEXOS 44

ÍNDICE DE TABLAS Tabla 1: Dimensiones de la brecha digital 19 Tabla 2:
Determinantes socioeconómicos de acceso, uso del TIC por parte de los
estudiantes en tiempos de confinamiento 20 Tabla 3: Determinantes
sociodemográficos de acceso, uso del TIC por parte de los estudiantes en
tiempos de confinamiento 21 Tabla 4: Determinantes geográficos de uso
del TIC por parte de los estudiantes en tiempos de confinamiento 22
Tabla 5: Variables y estadísticos 23 Tabla 6: Nivel de educación 28
Tabla 7: Gestión de la escuela 28 Tabla 8: Ámbito geográfico - gestión
de la escuela 29 Tabla 9: Ámbito geográfico - Acceso de Internet y Uso
de Internet 29 Tabla 10: Muestra por departamentos 33

ÍNDICE DE GRÁFICOS Gráfico 1: Porcentaje de países en los que se
implementa cada acción 25 Gráfico 2: Presupuesto Institucional
Modificado (PIM) asignado a educación como porcentaje del PBI 27

\begin{enumerate}
\def\labelenumi{\arabic{enumi}.}
\item
  INTRODUCCIÓN El pasado 4 de marzo del 2020, se registró en el Perú el
  primer caso de coronavirus. El 16 de marzo, casi dos semanas después,
  se impuso en el país una serie de medidas restrictivas de aislamiento
  social con el fin de disminuir la propagación del número de
  contagiados por COVID-19. Como resultado, esta nueva dinámica ha
  reconfigurado las rutinas diarias en diversos ámbitos, siendo el
  ámbito educativo uno de los más afectados. En este, las instituciones
  educativas han hecho uso de diversos medios tecnológicos con el fin de
  que los 8 millones1 de estudiantes de la educación básica regular
  continúen su aprendizaje. Así, el Estado mediante el Ministerio de
  Educación (MINEDU) desarrolló la plataforma ``Aprendo en Casa'' que
  fue transmitida a nivel nacional a través de radio, televisión e
  internet, mientras que en las escuelas privadas las clases se
  transmiten por medio de otras plataformas digitales. Sin embargo, este
  paliativo se ha transformado en una nueva manera de exclusión, ya que
  hay factores que condicionan el acceso a la educación en línea,
  acentuando la brecha digital. Clafin (2000), define esta brecha como
  ``la separación que existe entre las personas que pueden y usan las
  tecnologías de la información como una parte rutinaria de su vida
  diaria y aquéllas que no lo hacen''. Asimismo, como menciona Atuesta,
  Gonzales, Zea (1997) \& Castells (2006) en Grisales (2011) las
  Tecnologías de la Información y Comunicación (TIC) es
  {[}1{]}{[}2{]}definida como los: ``recursos tecnológicos que generan
  una sinergia comunicativa sin precedentes: palabra escrita; registros
  orales y visuales; dispositivos masivos de almacenaje con capacidades
  de ordenar, organizar y transformar información; dispositivos potentes
  de transmisión y comunicación; {[}…{]} que nos brindan disponibilidad
  casi universal de la información, y posibilitan la desaparición de los
  condicionantes de tiempo y espacio'' (Grisales, 2011, p.~6).

\begin{verbatim}
La definición de brecha digital se emplea tanto entre países como dentro de un mismo país; en ambos casos, la situación de Perú es preocupante. Si bien es cierto, durante la última década, la penetración de internet se ha ido incrementando a nivel mundial gracias a los desarrollos tecnológicos y la sociedad de información, a finales del 2020, Perú se posicionó en el puesto 8, luego de Chile, Uruguay, Argentina, Venezuela, Paraguay, Brasil y Colombia, en ese orden, posiciones respecto a la penetración de internet2. Ahora bien, dentro del país, solo el 67% de la población accedía a este servicio. Este porcentaje resulta ilusorio pues es necesario tener en cuenta la heterogeneidad del país. De modo que, el escenario resulta más desalentador si se hace este análisis por zonas geográficas del país. En las zonas urbanas, el 73% de la población tiene acceso a la red, mientras que, en las zonas rurales, un 27% accede a este servicio3. 
Además, la brecha digital tiene tres dimensiones - brecha de acceso, brecha de uso y brecha de apropiación. La primera dimensión está relacionada a la accesibilidad o no de las TIC; la brecha de uso está referida a cómo se usa las TIC y la brecha de apropiación está relacionada a para qué se usa o con qué fin se usan las TIC4. 
En la región de América Latina y el Caribe, solo 65% de la población que estudia tiene acceso a una computadora, mientras que 46,5% posee una Tablet; asimismo, en esta dimensión de acceso, no solo se presenta sobre infraestructura o accesibilidad a las TIC, sino que también se debe considerar la calidad de la cobertura que permita a los estudiantes estar conectados. Estas dificultades de acceso que enfrentan los educandos también lo hacen los maestros. Sumado a eso, por el lado de la dimensión de uso, la falta de competencias en capacidades tecnológicas o la comprensión de cómo utilizar las TIC por parte de los algunos maestros los apartan del uso de las herramientas tecnológicas; es decir, son excluidos digitales o analfabetos digitales pues a pesar de tener el acceso a las TIC no saben cómo utilizarlas. Por último, la dimensión de apropiación es la capacidad de saber sacar provecho de las TIC. Por ejemplo, si se utiliza el tiempo en red en videojuegos o para uso específicos de tareas escolares.
En este sentido, poder estudiar se vuelve un privilegio de cierta parte de la población, creando así una brecha mucho más amplia porque hay un sector de la población que está restringida de avanzar con sus actividades en este nuevo paradigma, quitando así uno de los derechos humanos fundamentales – derecho a la educación, siendo el sector de educación uno de los más afectados. El MINEDU estima que, 300 mil escolares de la matrícula escolar no han participado del sistema educativo ya sea por las necesidades de ir a trabajar y/o las dificultades para conectarse a las clases virtuales (Gestión, 2020). Asimismo, el MINEDU señala que 337,870 estudiantes de educación básica regular se trasladaron de instituciones privadas a instituciones públicas (MINEDU, 2020). Como señala Schmelkes (2020) este confinamiento podría convertir la brecha digital en una brecha de aprendizaje.
Asimismo, la emergencia sanitaria como consecuencia del COVID-19 ha sido motor para poder visibilizar y concientizar las desigualdades y carencias del sistema que ya existían en el país. A pesar de ello, este nuevo contexto muestra oportunidades para desarrollar soluciones de manera de acortar las brechas digitales con miras a un país con recursos tecnológicos mejor distribuidos. 
Dado el entorno actual, el objetivo de esta investigación es comprender los determinantes de la brecha digital en los estudiantes de educación básica regular en la educación no presencial provocada por el Covid-19, tomando en cuenta las dimensiones de acceso, uso y apropiación de internet. Para responder al objetivo general se proponen los siguientes objetivos específicos[3]:
\end{verbatim}

  • Analizar si los estudiantes cuentan con los recursos tecnológicos
  suficientes para continuar sus estudios. • Examinar si las habilidades
  y conocimientos son suficientes para implementar las recomendaciones
  en las escuelas en relación al uso de las TIC. • Estudiar si hay
  disparidad entre las variables analizadas, en los {[}4{]}medios que
  poseen y el aprovechamiento a la hora de su utilización. De modo que
  cada uno de estos objetivos, está relacionado a una dimensión
  específica de la brecha digital. Cabe mencionar que, la literatura
  sobre el tema es escasa. En particular, hay pocos análisis
  cuantitativos que midan la brecha digital en los estudiantes. De
  hecho, los estudios actuales revisados sobre el tema se enfocan en
  describir a la población analizada{[}5{]}. Asimismo, estos estudios
  nacionales e internacionales no toman en cuenta el factor pandemia.
  Esta investigación busca llenar este vacío, de manera que, esta
  investigación considera importante estudiar la brecha digital en los
  estudiantes en tiempos de confinamiento, tomando en cuenta un enfoque
  de acceso, uso y apropiación de internet. La investigación utilizará
  un enfoque cuantitativo, utilizando como principal fuente de
  información la Encuesta Nacional de Hogares - 2020 (ENAHO). La cual
  cuenta con datos de individuos sobre el acceso a dispositivos
  tecnológicos en el Perú. De esta forma, para el caso del primer
  objetivo específico se hace un modelo probit, en el cual la variable
  dependiente es el acceso a Internet. Para responder al segundo
  objetivo y al tercer objetivo, se realiza una relación causal entre
  las variables {[}6{]}{[}7{]}por medio de un modelo probit, en el cual
  la variable dependiente es uso de internet. En contextos normales,
  esta brecha digital forma parte de una de las desigualdades existentes
  en Perú, y en tiempos de pandemia, las consecuencias a largo plazo
  pueden ser aún más perjudiciales. Sin embargo, en este contexto donde
  parte de los estudiantes en todos los departamentos se están quedando
  fuera de este sistema digital, queda claro que de alguna manera
  afectará su posibilidad de producir ingresos a futuro, ya que el
  estudiantado se ve limitado de aprender, de tener contactos en el
  sistema. Según la Organización de las Naciones Unidas para la
  Educación, la Ciencia y la Cultura (UNESCO, siglas en inglés), la
  educación es un derecho para todo individuo pues genera ``las
  capacidades y conocimientos críticos necesarios para convertirnos en
  ciudadanos empoderados, capaces de adaptarse al cambio y contribuir a
  la sociedad'' (UNESCO, 2014). Es así que determinar la brecha digital
  en los estudiantes {[}8{]}{[}9{]}en los estudiantes durante el 2020 se
  convierte en una herramienta esencial para determinar si han logrado
  una asimilación adecuada de la educación en línea en términos
  tecnológicos, evitando su exclusión del sistema educativo. Esto
  permite trazar estrategias encaminadas hacia la equidad digital,
  priorizando a los estudiantes de EBR con recursos tecnológicos
  limitados. El texto está estructurado de la siguiente manera. En la
  sección 2, se presenta la revisión de literatura donde se definirá, el
  concepto de sociedad de conocimiento y la relación con la brecha
  digital, luego se detallará el concepto de brecha digital, de esta
  manera se analizarán las dimensiones de esta; posteriormente, se
  expondrá el stock actual de conocimientos de los temas las relevantes
  en la literatura analizada. Luego se planteará el marco teórico a
  utilizar en esta investigación, donde se presenta el modelo teórico.
  Los hechos estilizados se presentan en la sección 4. Posteriormente,
  se muestran las hipótesis de la investigación. El modelo econométrico
  y las variables a analizar se presentan en la sección 6. {[}10{]}En la
  siguiente sección, se muestran los resultados de las estimaciones. Por
  último, se muestran las conclusiones de la investigación.
\item
  REVISIÓN DE LITERATURA En los últimos veinte años, la literatura sobre
  la brecha digital ha aumentado, en particular en relación a la brecha
  de uso y de aprovechamiento. Esto como consecuencia del incremento de
  usos de las TIC alrededor del mundo. En la actualidad, debido a la
  coyuntura, la brecha digital es un tema que se ha visibilizado aún
  más. Sin embargo, en el caso específico peruano, existen escasas
  investigaciones sobre la brecha digital en el sector de educación y
  ninguna toma en cuenta la pandemia. En esta sección, se muestra la
  revisión de literatura nacional e internacional que aporta a estudiar
  y examinar el tema en cuestión. Esta sección está dividida en cuatro
  secciones. Primero, se precisa la definición de sociedad de
  información, que ayudará a una mejor comprensión de los temas
  relevantes. Segundo, se detalla la definición de brecha digital, y se
  muestran las dimensiones de esta. En tercer lugar, a partir de las
  dimensiones de la brecha, se estudia los determinantes de acceso, uso
  y apropiación de las TICS. Finalmente, se realiza un balance de la
  revisión de literatura. 2.1. SOCIEDAD DE LA INFORMACIÓN Y BRECHA
  DIGITAL: BENEFICIOS Y BARRERAS La sociedad de la información es
  entendida como ``el espacio social altamente dinámico, abierto,
  globalizado y tecnologizado, donde el conjunto de relaciones sociales
  (acción e interacción de los individuos, procesos de producción
  material y espiritual) se apoyan y realizan a través de la
  información'' (Garduño, 2004, 4). El siglo XXI presenta una cadena de
  cambios, oportunidades y retos para las personas gracias a la
  presencia de las TIC. Según Castells, ``La importancia que estas
  tecnologías tienen para la sociedad de la información es el
  equivalente a la que tuvo la fábrica durante la revolución
  industrial'' (Márquez et al., 2016: 95).\\
  En este sentido, en la actualidad, el principal recurso de desarrollo
  en la sociedad es la información que puede ser transmitida de manera
  descentralizada a todas las partes de la sociedad a través de los
  medios digitales, de esta forma estar conectados con el mundo; además
  ha contribuido en el cambio de estructura de las actividades sociales.
  Esto debido a que, en ``la sociedad de la información el ciberespacio
  se convierte en el escenario de comunicación interactivo y
  comunitario'' (Garduño, 2004: 4), donde las personas pueden acceder a
  la información a un clic de distancia mediante dispositivos digitales.
  La sociedad de la información es aquella donde la principal fuente de
  productividad y competitividad es la información (Castells, 2000).
  Asimismo, se hace referencia de que esta sociedad de la información o
  Estado-red este interconectada mediante las TIC (Castells,2000). De
  esta manera, se produce un tipo de red interconectada por las
  tecnologías de información, de modo que el conocimiento puede ser
  adquirido por cualquiera que pertenezca a esta sociedad. Sin embargo,
  ``uno de los retos de la sociedad en red es advertir cómo los procesos
  de exclusión se reeditan, ahora en función al acceso y dominio de los
  nuevos códigos sociales, laborales, económicos y culturales {[}…{]}. Y
  esta situación contribuye a la generación de un nuevo tipo de brecha
  social: la divisoria digital'' (Steinberg, 2013: 88-89). De este modo,
  quienes no están conectados a este sistema quedan excluidos de los
  beneficios de esta sociedad- red. Es decir, la sociedad de información
  presenta una serie de ventajas, pero también crea barreras a quienes
  no puedan estar en este espacio de la sociedad. En este marco, surge
  la necesidad de reducir la brecha digital dado sus efectos
  restrictivos. Además, hay que tener en cuenta que la mejora en
  infraestructuras tecnológicas es un requisito necesario, más no
  suficiente para acceder a los beneficios de la sociedad digital
  (Lera-López, Hernández \& Blanco, 2003). Asimismo, Bernabeu (1997)
  señala la importancia de transformar la educación tradicional en
  respuesta al progreso de la sociedad y al nuevo concepto de educación,
  adaptándose a las nuevas necesidades emergentes y a las cambiantes
  formas de desarrollo. 2.2. DEFINICIÓN DE BRECHA DIGITAL Y SUS
  DIMENSIONES Sunkel{[}11{]} et. Al. (2011) muestra que existen dos
  tipos de brecha, la brecha internacional es la que se da por las
  diferencias de acceso e infraestructura entre países y regiones del
  mundo; mientras que la brecha interna se hace presente entre distintos
  grupos sociales dentro de un país. Este estudio tomará la definición
  de brecha interna de Sunkel et Al. Asimismo, James (2001)
  conceptualiza la brecha digital a nivel de individuo utilizando el
  número absoluto de personas que usan Internet y de personas que hacen
  uso de teléfonos móviles de países en desarrollados con respecto a
  países desarrollados. Según Clafin (2000), ``la Brecha Digital es la
  separación que existe entre las personas que pueden y usan las
  tecnologías de la información como una parte rutinaria de su vida
  diaria y aquéllas que no lo hacen''. Una definición parecida también
  lo propone Tello, en palabras de este, ``es una línea que separa a las
  personas que ya se comunican y coordinan actividades mediante redes
  digitales de quienes aún no han alcanzado este estado avanzado de
  desarrollo'' (Tello, 2007: 3). Mientras que Castells (2001) y UIT
  (2009) aseguran que la brecha digital es la diferencia entre los que
  tienen y lo que no tienen Internet. Algunos escritores resaltan esta
  disparidad en la habilidad que tienen los consumidores en saber usar
  las TIC y obtener beneficio de estas. Por ejemplo, Monge \& Hewiett
  (2004) indica que la brecha se da a consecuencia del acceso desigual
  de los usuarios de las TIC, y sus habilidades para poder utilizar
  estas herramientas, dependiendo del aprovechamiento de uso de estas.
  En cuanto a las definiciones de la brecha digital hechas por las
  instituciones se ha encontrado que estas las definen tomando en cuenta
  el ámbito tecnológico. Según World Resources Institute (s.f) la brecha
  digital se refiere a la brecha entre quienes tienen acceso a
  información computarizada e Internet y quienes no. Asimismo, la
  Organización de las Naciones Unidas para la Educación, la Ciencia y la
  Cultura – UNESCO (2005) define a la brecha digital como una brecha
  cognitiva que genera desigualdades entre países en la producción y
  participación de conocimientos. Por último, la Comisión Económica para
  América Latina y el Caribe asegura que la brecha digital tiene
  dimensiones, ``extensión (acceso) y profundidad (calidad de acceso)''
  (CEPAL, 2008). Con el transcurso de los años, la brecha digital está
  tomando diversos matices en{[}12{]}{[}13{]} la definición. Por lo
  tanto, para una comprensión más precisa de esta definición de brecha
  es necesario detallar las dimensiones de esta. La mayoría de las
  definiciones anteriores se basan en acceso e infraestructura. Sin
  embargo, la disposición de las computadoras e Internet en países
  desarrollados, ha impulsado que este término aborde aspectos cada vez
  más complejos5. En la literatura se encuentra que hay tres dimensiones
  y estas se presentan de una manera jerárquica. De esta manera Crovi
  (2008) menciona que existen tres dimensiones de la brecha digital,
  brecha de acceso, brecha de uso y brecha de apropiación. La primera
  dimensión se define como ruptura entre los que tienen entrada a la
  infraestructura técnica o infraestructura de telecomunicación y los
  que están apartados de estas. Sin embargo, Selwyn (2004) menciona que
  no solo se trata de poder adquisitivo y acceso físico nombrados por
  este autor como acceso formal. Sino que se debe tener en cuenta el
  acceso efectivo, es decir se debe considerar más allá de la capacidad
  y distribución del acceso. Por ejemplo, para acceder a la información
  desde un celular o una computadora en el hogar no es lo mismo que
  acceder a la información mediante recursos de la escuela, cafés o
  cabinas de internet. Entonces, en este primer nivel comprende tanto el
  acceso formal como el acceso efectivo a las TIC. Se debe tener en
  cuenta que el acceso a las TIC no necesariamente involucra saber
  utilizar estas TIC. La segunda dimensión hace referencia a quienes
  hacen uso frecuente de las TIC de las que no lo hacen, ya sea por
  desconocimiento, desinterés o porque desconocen los beneficios de
  estas. Asimismo, Van Dijk (2017) menciona que luego de obtener el
  acceso a las TIC, es necesario tener habilidades para utilizarlas,
  añade también que en esta dimensión es importante la frecuencia, el
  número de horas que se utilizan las TIC. Esta dimensión de la brecha
  se da como consecuencia de los hábitos educativos o los hábitos
  laborales que motiva a los que hacen uso de las TIC pues de alguna
  manera les aporta algún valor. Por último, la brecha de apropiación es
  la separación de los usuarios que le dan usos específicos a las TIC y
  los que hacen usos triviales de las TIC. Mientras que el primer grupo
  le da un uso ``valioso'', de tal forma de poder generar cambios
  significativos en su entorno, los otros, no sacan provecho de estas
  herramientas (Márquez et al., 2016). En esta dimensión no solo es
  necesario tener acceso a las TIC y saber utilizarlas, tampoco se toma
  en cuenta la calidad ni la cantidad, sino la apropiación recae en el
  uso significativo, el cual debe ser útil al individuo. Para la Peres
  \& Hilbert, plantean la incorporación en el currículo objetivos y
  contenidos relacionados a la integración de las TIC (2009). En el
  Perú, hay que rescatar dos investigaciones las cuales concuerdan con
  Van Dijk (2017) y Selwyn (2004). Primero, Tello en el 2018 en su
  investigación de productividad de emprendimiento informales analiza
  estas dimensiones como acceso o no de los usuarios de TIC, el manejo
  de los dispositivos tecnológicos y los resultados tangibles de estos
  (2018). Segundo, en Barrantes \& Vargas (2017) estas dimensiones están
  estudiadas como brecha de primer orden, brecha de segundo orden, y
  brecha de tercer orden. 2.3. DETERMINANTES DE ACCESO, USO DE LAS TIC
  POR PARTE DE LOS ESTUDIANTES EN TIEMPOS DE
  CONFINAMIENTO{[}14{]}{[}15{]} Respecto a los determinantes del acceso,
  uso de TIC, concretamente del Internet, por parte de los estudiantes
  se ha encontrado que hay variables que influyen en estas dimensiones.
  Se ha agrupado en tres tipos de determinantes, determinantes
  socioeconómicos (ingreso, nivel socioeconómico y costos);
  determinantes sociodemográficos (sexo, edad, estructura familiar);
  determinantes geográficos (área urbana/área rural). En esta parte se
  tomará en cuenta el factor confinamiento por el Covid-19. 2.3.1.
  DETERMINANTES SOCIOECONÓMICOS Y EDUCATIVOS Respecto{[}16{]} a los
  determinantes económicos se toman en cuenta los ingresos de los
  familiares. En un estudio hecho por James a los habitantes de la India
  encuentra que hay una relación entre los ingresos económicos y el
  acceso de internet, asimismo menciona que es posible cerrar esta
  brecha utilizando TIC de bajo costo (James, 2001). Asimismo, el Banco
  Mundial (1999) confirma que existe una relación entre las condiciones
  económicas y el acceso a las TIC, esto también ayuda al progreso
  económico de los países. En la misma línea, Duplaga (2017) muestra que
  un mayor nivel de ingresos, mayor es el uso de tecnologías. Entonces,
  tanto en países y personas con mayores ingresos económicos mayor será
  su acceso y uso a las TIC, caso contrario, el acceso y uso a estas
  será menor. En cuanto a nivel socioeconómico, Sádaba (2010) en su
  estudio a población de española muestra que el 45,3\% de los usuarios
  de internet procede de una clase social media-media, mientras que el
  36,7\% es de clase social media-alta y 15,7\% de una clase social
  media-baja. Asimismo, por el lado de los usos diferenciados, se
  encuentra que los estudiantes de menores recursos económicos, hacen
  usos consumistas y pasivos o superficiales (Van Deursen \& van Dijk,
  2013), mientras que el estudiantado de nivel socioeconómico alto, al
  poseer mayor tiempo disponible en la red puede realizar actividades
  culturales y de ocio (DiMaggio et al.~,2004); sumado a esto, el
  entorno familiar ayuda a controlar y orientar a estos estudiantes para
  que hagan usos más selectivos, formativos (Fernández, 2020),
  habilidades de pensamiento crítico (Hohlfeld y col.~, 2008). Por otro
  lado, otros autores, mencionan que la decisión de acceso a las TIC
  está sujeto al costo de la computadora y del internet. Para Murthy \&
  Soleimani (2015) se tiene que poder pagar los dispositivos como
  computadoras, tablets, celulares, etc. Para poder usar Internet.
  Además, Warschauer (2004) muestra que para poder conectarse a la red
  hay que tener además de una computadora se debe considerar el
  mantenimiento al equipo y el costo de conexión. Entonces, la capacidad
  adquisitiva limita la compra de los dispositivos y uso de estos (Claro
  et al., 2011). Kiiski \& Pohjola. (2002) en su estudio a 23 países
  miembros de la OCDE encuentran que una disminución de los costos de
  acceso a Internet, aumentaría el número de usuarios de computadoras.
  Asimismo, los costos de internet están determinados por la competencia
  entre los proveedores (Grubesic 2008). Cabe mencionar que, debido al
  progreso tecnológicos la compra de dispositivos como laptop o
  computadora resultan asequibles para ciertos sectores de la economía.
\end{enumerate}

Tabla 1: Determinantes socioeconómicos de acceso, uso del TIC por parte
de los estudiantes en tiempos de confinamiento

Determinantes socioeconómicosCaracterísticasPrincipales
referenciasIngresosAcceso: El ingreso de los padres afecta positivamente
en la adopción y uso Internet.James (2001), Banco Mundial (1999), Sádaba
(2010).Uso: Un mayor ingreso tiene relación positiva con mayor uso de
tecnologías. Duplaga (2017) Nivel SocioeconómicoAcceso: A mayor nivel
socioeconómico mayor acceso a Internet.Sádaba (2010)Uso: Un mayor nivel
socioeconómico tiene una relación positiva con el uso de Internet.Sádaba
(2010)Apropiación: Un nivel socioeconómico alto realiza actividades
formativas y selectivas.Van Deursen \& van Dijk, (2013), DiMaggio et
al.~(2004), Fernández (2020) y Hohlfeld y col., (2008). CostosAcceso:
los costos que implican estar conectados a las redes imitan el acceso a
las TIC.Murthy \& Soleimani (2015), Warschauer (2004), Claro et al.,
(2011) y Kiiski \& Pohjola. (2002) y Grubesic (2008).Fuente:
Elaboración: propia. En base a varios (indicados en la tabla).

2.3.2. DETERMINANTES SOCIODEMOGRÁFICOS En cuanto a determinantes
sociodemográficos, se encuentran literatura empírica principalmente
sobre las variables sexo, edad y la estructura familiar. En primer
lugar, Rodríguez (2006) señala que en ``las sociedades donde la
discriminación y marginación son culturalmente aceptadas, las mujeres no
tienen acceso a Internet o en el mejor de los casos tienen una mínima
posibilidad de utilizar la red y recibir información''. Sádaba (2010)
muestra que en el acceso y uso de las TIC existen grandes diferencias en
el uso de internet, del móvil y el ordenador, que dependen del sexo del
usuario. El porcentaje de uso del móvil es igual en ambos (90\%),
mientras que en el caso de uso de ordenador los hombres hacen mayor uso
de estos. Asimismo, el uso del internet es mayormente por hombres. Por
último, los hombres compran más internet 31,6\% vs 23,4\%. Losh (2003)
explica que el menor consumo de Internet por parte de las mujeres puede
estar explicado por el nivel educativo, ingreso y las responsabilidades
dentro del hogar. Además, Brynin (2006) indica que tanto hombres como
mujeres hacen uso de las TIC en la misma proporción. Sin embargo, el
estudio de Cuevas \& Alvarez (2009) encuentra diferencias en los
patrones de uso, pues los hombres hacen mayor uso y con mayor facilidad
del computador. En el estudio de Haugland (2000) detalla que, a partir
del cuarto grado, los niños hacen mayor uso de las computadoras que las
niñas. Es cuando estas entran a nivel secundario que esta brecha entre
hombre y mujeres se amplía (Gehring, 2001). En la investigación de
Calderón a estudiantes de España encuentra que el nivel de formación
digital de las mujeres es menor que en hombres (Calderón, 2019).
Jiménez, Vega \& Vico (2016) mostraron en su estudio de habilidades
técnicas de las mujeres que entre sus principales actividades se
encuentran video-llamadas, creación de contenido en redes sociales, en
un segundo realizan actividades de manejo de programas, conexión o
instalación de dispositivos, y programación. En contraposición, Castaño,
Duart \& Sancho (2012) muestran que las estudiantes universitarias de
España utilizan el internet con fines académicos; mientras que los
estudiantes hombres españoles lo hacen para juego online. Asimismo, Chen
\& Wellman (2004), Kuttan \& Peters (2003) y Kelley-Salinas (2000)
muestran que esta brecha de uso ha disminuido con el paso de los años,
asimismo muestran que en países desarrollados esta brecha es menor.
Según la OECD, el factor género a medida que aumenta el uso de Internet
se vuelve un factor irrelevante (OECD, 2007). En segundo lugar, la edad
es considerada un factor esencial que determina la brecha digital.
Rodríguez (2006) menciona que hay una resistencia al uso de las TIC por
aquellas personas que eran mayores de edad antes de la revolución del
internet en la década de 90. Prensky (2001) en su análisis de brecha
digital por edades, menciona dos grupos importantes. Primero, los
nativos digitales son individuos que hicieron usos de dispositivos
digitales desde temprana edad. Segundo, el migrante digital es definido
como aquella persona que nació antes del boom de las nuevas tecnologías.
De esta forma, quienes nacieron junto a los avances del internet son los
llamados ``nativos digitales'', estos utilizan la red con gran facilidad
y quienes nacieron antes de estos avances son los ``migrantes
digitales''. Asimismo, Kuttan \& Peters (2003) hacen un análisis por
grupo de edades y encuentra que las personas entre las edades de 18 y 49
años son los usuarios más frecuentes y que los individuos mayores a 50
años hacen menor uso de internet. En la misma línea, Sádaba (2010)
muestra que la edad de los internautas que usan las TIC se encuentra
entre 14 y 34 años con un 50,8\%, mientras los internautas entre 35 y 54
años representan el 36,4\% y los internautas mayores de 55 años
representan el 12,8\%. Es decir, hay una relación inversa entre la edad
de la persona y el uso del internet. Cueto, León \& Felipe (2020)
encuentran que mientras más temprana sea la edad de uso de las TIC,
mayor frecuencia de uso de computadoras e Internet y mayores habilidades
digitales. Además, el uso de Internet y computadora tienen una relación
negativa con la edad, debido a que mientras más años tenga menor la
motivación de acceder y usar las TIC (Duplaga, 2017; Lera-López et al.,
2011). Los estudiantes de últimos grados de secundaria hacen mayor uso
del Internet (Quiroz, 2014). En cuanto al uso que se les da al Internet,
se encuentra que quienes hacen mayores usos de ocio e interacción son
los jóvenes (Castell et al., 2007). Sin embargo, hay que tener en cuenta
que no necesariamente los estudiantes nacidos en la era digital cuentan
con habilidades digitales necesarias para la sociedad (Álvarez-Sigüenza,
2019). Finalmente, la estructura familiar contribuye en el acceso y uso
del Internet. Según Mills \& Whitacre (2003), la decisión de adopción de
internet tiene una influencia positiva cuando hay niños en las familias.
Sin embargo, McKeown, Noce \& Czerny (2007) sugieren que en efecto hay
una relación positiva entre hogar con niño y conexión a Internet, pero
esto no necesariamente significa que los adultos del hogar también los
utilicen.

Tabla 2: Determinantes sociodemográficos de acceso, uso del TIC por
parte de los estudiantes en tiempos de confinamiento

Determinantes sociodemográficos Características:Referencias:SexoAcceso:
Las mujeres tienen un menor acceso al Internet.Rodríguez (2006), Sádaba
(2010) y Losh (2003) Uso: Los hombres tienen mejores habilidades
digitales. Sádaba (2010), Brynin (2006), Cuevas \& Alvarez (2009),
Gehring, (2001), Haugland (2000) y Calderón (2019). Apropiación: No hay
un consenso con respecto al uso valioso con respecto a la variable sexo.
Jiménez, Vega \& Vico (2016) Castaño, Duart \& Sancho (2012) Edad Uso:
la población que más usa Internet son los llamados ``nativos
digitales'', mientras más temprana sea la edad de uso de las TIC mayores
habilidades digitales.

Rodríguez (2006), Prensky (2001), Kuttan \& Peters (2003), Duplaga,
2017, Lera-López et al., 2011)., Álvarez-Sigüenza y Sádaba
(2010)Apropiación: Los jóvenes son quienes hacen mayores usos de
interacción en actividades de ocio.

(Castell et al., 2007).Estructura familiarAcceso: tener niños en casa
aumenta la probabilidad de estar conectado a Internet. Mills \& Whitacre
(2003) y McKeown , Noce \& Czerny (2007)Fuente: Elaboración: propia. En
base a varios (indicados en la tabla).

2.3.3. DETERMINANTES GEOGRÁFICOS En cuanto a los determinantes
geográficos, los estudios se centran en las disparidades con respecto a
la ubicación del estudiante; mientras que la infraestructura del hogar
está relacionada con los recursos tecnológicos de estas. Por un lado,
Rodríguez (2006) muestra que hay una diferenciación en los servicios de
TIC entre zonas rurales y zonas urbanas, estas últimas tienen mejor
acceso a los servicios. Asimismo, en el estudio de Sunkel et al.~(2011),
para países de Latinoamérica y El Caribe, muestra que hay una escasa
cobertura e infraestructura de los servicios de telecomunicación y esto
se ve reflejado en un déficit de equipamiento en las familias de zonas
rurales. Por otro lado, en la investigación de Riddlesden \& Singleton
(2014) revela disparidades en la velocidad de Internet entre las zonas
rurales y urbanas. Estas diferencias pueden atribuirse a los costos
asociados con la implementación de tecnología. Estas disparidades pueden
ser explicadas, por los costos de implantación de tecnología. Schneir \&
Xiong (2016) comprueban que las inversiones en áreas rurales o con poca
densidad de población son desalentadas por el bajo retorno de la
inversión.\\
Antes de la pandemia, estas desigualdades podrían solucionarse con
infraestructura gracias a los computadores de las escuelas, cabinas de
internet o cafés. Por un lado, Rodríguez \& Sandoval (2017) en un
estudio a estudiantes chilenos encuentran que la escuela es un ``gran
democratizador del acceso a las TIC, potenciando las posibilidades de
adquisición de mayor capital social y cultural por parte de los
estudiantes''. Sin embargo, Sunkel (2006) asegura que estos medios se
ven restringidos por el tiempo que disponen los estudiantes para hacer
uso de estas herramientas. Por otro lado, en cuanto a las cabinas
públicas, Teresa Quiroz (2014), en su estudio a estudiantes de
secundaria de Perú, encuentra que el casi el 40\% de los estudiantes
hace uso de las cabinas públicas por razones escolares, o de
entretenimiento, mientras que el 38,1 \% y 58,7\% lo hacen en áreas
urbanas y áreas rurales respectivamente. En la misma línea, Zhao et
al.~(2010) muestra que el la duración y frecuencia de uso de TIC se
deben tomar en cuenta para el aprovechamiento de TIC; además del lugar
donde se realiza la conexión (escuela, hogar o cibercafé). Rhee \& Kim
(2004) demuestran que estas conexiones alternas en trabajo, café o
cabinas públicas contribuyen a la reducción de desiguales tecnológicas.
Ahora bien, dado el contexto, los recursos tecnológicos del trabajo,
café o cabinas públicas ya no son opción de acceso y uso de internet por
el confinamiento social. Entre todos estos determinantes, los modelos
cuantitativos empleados en la literatura para llevar a cabo las
estimaciones son principalmente el probit o el logit. Además, en algunos
de estos estudios se llevaron a cabo encuestas y/o entrevistas propias
para recopilar los datos.

Tabla 4: Determinantes geográficos de uso del TIC por parte de los
estudiantes en tiempos de confinamiento

Determinantes geográficos CaracterísticasReferenciasÁrea
GeográficaAcceso: La baja densidad de población y los altos costos de
implementación desalientan las inversiones privadas. Las áreas rurales
tienen un déficit de infraestructura. Gallado (2006), Sunkel et
al.~(2011), Riddlesden \& Singleton (2014), Rendon Schneir \& Xiong
(2016), Rodríguez \& Sandoval (2015), Teresa Quiroz (2014), Zhao et
al.~(2010), Rhee \& Kim (2004). Fuente: Elaboración: propia. En base a
varios (indicados en la tabla).

2.4. BALANCE DE LA REVISIÓN DE LITERATURA{[}17{]}{[}18{]} En resumen, en
esta parte se ha hecho hincapié en cuatro aspectos relevantes para este
trabajo: sociedad de información, brecha digital y sus dimensiones; y
determinantes de acceso, uso y apropiación, así como la teoría del
modelo de los cuatro tipos de acceso en la apropiación de la tecnología
digital propuesta por Van-Dijk. Respecto a la definición de sociedad de
información se encuentra que está presenta diversos beneficios para los
que se encuentran dentro de esta. Sin embargo, pone barreras para
quienes no; es decir los excluye; creándose así una brecha. Esta brecha
digital es definida como desigualdad en términos tecnológicos. Para una
mejor comprensión de la brecha digital se estudió sus dimensiones. La
literatura analiza tres dimensiones: (i) brecha de acceso, referida a la
desigualdad entre los que pueden acceder a dispositivos digitales y
aquellos que no, considerando una infraestructura adecuada; (ii) brecha
de uso, referida a si la persona hace uso o no de las TIC, por diversos
motivos como la motivación y/o falta de interés; (iii) brecha de
apropiación, referida a la diferencia entre quienes hacen uso productivo
de las TIC de quienes hacen uso básico de estos. Respecto a la
literatura empírica internacional sobre los determinantes, el nivel de
ingresos de los familiares, un nivel socioeconómico alto, la edad (si es
mayor de 15 años) y el nivel educativo (si está en secundaria) influyen
positivamente en el acceso y uso de Internet por parte de los
estudiantes. Mientras que las variables como lugar geográfico (si vive
en zona rural), costo influyen negativamente en el acceso y uso de
internet por parte de los estudiantes. Por último, la variable sexo (si
es mujer) no da una relación precisa. Por último, la literatura empírica
internacional es analizada para países de América Latina, Europa, Asia y
África; en cuanto a literatura nacional, se encuentra un estudio de
Quiroz escrito en el 2014, el cual analiza los alcances de las TIC en
las aulas. Su análisis es hecho para los años 2006 a 2012. No se ha
hallado literatura peruana que tome en consideración el factor pandemia,
en consecuencia, se considera importante la investigación.

\begin{enumerate}
\def\labelenumi{\arabic{enumi}.}
\setcounter{enumi}{2}
\item
  MARCO TEÓRICO{[}19{]}{[}20{]} En esta sección se presenta primero el
  modelo teórico de la investigación. Luego se detalla las dimensiones
  de brecha digital. Finalmente, se analiza los determinantes de la
  brecha digital que se presentan en cada dimensión. 3.1. MODELO TEÓRICO
  En esta parte se sigue a Grazzi \& Vergara (2009) y Tello (2018) para
  realizar los análisis de acceso a Internet; mientras que para el
  análisis de uso a Internet se sigue a Madden \& Simpson (1997);
  Lera?López et al.~(2011); Mills \& Whitacre, (2003); Vicente \& López,
  (2008) y Tello (2018). Primero, en el caso específico de este análisis
  para acceder a las TIC se requiere acceso a Internet y acceso a un
  dispositivo con el cual conectarse (computadora, laptop, tablet,
  celular). Siguiendo a Grazzi \& Vergara (2009) el acceso a un
  dispositivo digital le generará una utilidad indirecta, en donde si la
  utilidad de tener el dispositivo es definida como (U\_(i,H)) y la
  utilidad de no tener una computadora (U\_(i,N)). En donde el individuo
  i elegirá tener el dispositivo si la utilidad (U\_(i,H))
  \textgreater{} (U\_(i,N)). Del mismo modo el individuo j elegirá tener
  acceso a internet, si la utilidad de tener internet (U\_(i,C)) es
  mayor a la utilidad de no tener internet (U\_(i,N)). Tanto la utilidad
  de tener un dispositivo con el cual conectarse a Internet como acceder
  a Internet es una función lineal de un conjunto de características
  socioeconómicas, sociodemográficas y geográficas del hogar. En este
  caso, la adopción de algún dispositivo digital es una condición
  preliminar para tener conexión de Internet en casa. El marco de
  referencia en el que se basará esta investigación en el caso de uso de
  las TIC es el de la teoría neoclásica del consumidor para relacionar
  la probabilidad de hacer una elección con cierto conjunto de
  comportamientos que reflejan las preferencias de los individuos
  (Madden \& Simpson, 1997; Lera?López et al.~2011; Mills \& Whitacre,
  2003; Vicente \& López, 2008). De modo, cada que la utilidad (U\_ij)
  representa a utilidad del individuo (i) que obtiene de la alternativa
  (j). {[}21{]}En este caso, j está relacionada al uso de internet
  (j=1), o al no uso de este servicio (j=0), la utilidad del uso de
  internet es U\_i1 mientras que la de no uso de internet es U\_i0. Al
  maximizar la utilidad de la alternativa j del individuo se obtiene
  U\_ij=U\_ij (p\_j, y\_i,s\_i, w\_i) Donde: ? p\_j: precios del
  servicio j ? y\_i: ingreso de la familia de individuo i ? s\_i:
  características observables del individuo i ? w\_i: factores no
  observables Maximizando la utilidad de uso de internet se tiene:
  U\_ij=U\_ij (p\_j, y\_i,s\_i, w\_i) En la función de utilidad se asume
  que la utilidad de uso de internet depende de las características del
  individuo. En particular, se investigará la influencia de la edad, el
  sexo, ubicación geográfica en el uso o no de internet. Entonces, el
  hogar i hará uso de internet si y solo si: U\_i1 (p\_j, y\_i,s\_i,
  w\_i )\textgreater U\_i0 (p\_j, y\_i,s\_i, w\_i) La probabilidad de
  que el individuo i haga uso del internet es: P\_i1=Prob?{[}U?\_i1
  (p\_j, y\_i,s\_i, w\_i )\textgreater U\_i0 (p\_j, y\_i,s\_i, w\_i){]}
  3.2. BRECHAS: ACCESO, USO E INTENSIDAD DE USO{[}22{]}

\begin{verbatim}
La brecha digital, que abarca diversas dimensiones, ha sido objeto de estudio en la literatura especializada. Estas dimensiones, a saber, brecha de acceso, brecha de uso y brecha de apropiación, revelan desigualdades profundas en el ámbito tecnológico. La brecha de acceso refiere a la disparidad entre aquellos que pueden acceder a dispositivos digitales y aquellos que no pueden hacerlo, considerando la disponibilidad de una infraestructura adecuada. Por otro lado, la brecha de uso se relaciona con la decisión de una persona de utilizar o no las Tecnologías de la Información y Comunicación (TIC), influenciada por diversos factores, como el nivel de motivación o la falta de interés. Finalmente, la brecha de apropiación señala la diferencia entre quienes hacen un uso productivo y enriquecedor de las TIC y aquellos que solo realizan un uso básico o superficial de estas tecnologías. La comprensión de estas dimensiones es esencial para abordar los desafíos que plantea la brecha digital y para diseñar estrategias efectivas de inclusión tecnológica.
\end{verbatim}
\end{enumerate}

Tabla 1: Dimensiones de la brecha digital

Brecha de accesoBrecha de usoBrecha de apropiaciónReferida a la
desigualdad entre los que pueden acceder a dispositivos digitales y
aquellos que no, considerando una infraestructura adecuada.Referida a si
la persona hace uso o no de las TIC, por diversos motivos como la
motivación y/o falta de interés.Referida a la diferencia entre quienes
hacen uso productivo de las TIC de quienes hacen uso básico de estos.
Fuente: Elaboración Propia.

\begin{enumerate}
\def\labelenumi{\arabic{enumi}.}
\setcounter{enumi}{3}
\tightlist
\item
  HECHOS ESTILIZADOS{[}23{]} En la presente sección se detallará, los
  estadísticos descriptivos, para ello se utilizará los datos extraídos
  de los módulos uno (Características de la Vivienda y del Hogar), dos
  (Características de los Miembros del Hogar) y tres (Educación) de la
  Encuesta Nacional de Hogares ENAHO (2020). Luego, se examina las
  políticas de los países de la región de América Latina y El Caribe.
  Finalmente, se analiza la situación actual del Perú. 4.1.
  DATOS{[}24{]} La muestra está compuesta por 13,237 estudiantes de
  diferentes niveles educativos. En la Tabla 5, se evidencia los
  principales estadísticos descriptivos de las variables.
\end{enumerate}

Tabla 5: Variables y estadísticos

VariableObsMediaStd.
Dev.MinMaxSociodemográficas~~~~~Edad13,23710.452144.221995318Sexo
(mujer=2, hombre=1)13,2371.4883280.499882612Idioma (castellano=1,
otro=0)13,2370.86492410.341817601Educación~~~~~Grado13,2372.1946060.708957913Centro
de estudios (estatal=1,
otro=2)13,2321.1262850.332182612Geográficas~~~~~Ámbito (urbano=1,
rural=0)13,2370.55133340.497376701Costa13,3410.34555130.475565501Sierra13,3410.36511510.481480501Selva13,3410.28933360.453470101Uso
de TIC~~~~~Conexión a Internet13,2660.25192220.434133201Uso
Internet11,2131.5466870.497837812Uso
computador5,0830.51918160.499681101Uso celular sin plan de
datos5,0830.39799330.489532201Uso celular con plan de
datos5,0830.15856780.365308501Fuente: ENAHO2019. Elaboración: propia.

\begin{verbatim}
   De acuerdo con la Tabla 5, las edades de los estudiantes están comprendidas entre 3 y 18 años; la edad promedio de los estudiantes es de 10 años. El 74,4% de la muestra es mujer y el 86% tiene como idioma principal el castellano. Asimismo, el 55% vive en zonas urbanas. El 36% de los estudiantes pertenece a la Sierra, el 34% proviene de la Costa y el 28 proviene de la Selva. En cuanto al centro de estudios, el 56% lleva clases en un centro estatal. Por último, solo el 25% de 13,226 estudiantes tienen acceso a Internet. El 51% usa un computador (N=5,083) y el 39% usa celular con un plan de datos y un 15% hace uso del celular sin un plan de datos.
\end{verbatim}

4.2. SITUACIÓN EN LA REGIÓN{[}25{]} 4.2.1. POLÍTICAS TIC EN LA EDUCACIÓN
La existencia de estas políticas en el sector educación son de gran
importancia para que las TIC den respuesta a los desafíos que se
presenten y posibilita un desarrollo de condiciones mínimas. Estas
condiciones mínimas son puestas en eLAC6 cuyo plan de acción a largo
plazo es que las TIC sean instrumento para el desarrollo económico y la
inclusión social para los países de América Latina y el Caribe. Hasta la
fecha se realizado cuatro versiones de ese plan: eLAC2007, eLAC2010,
eLAC2015 y elac2018. En el 2010, nueve países de la región publicaron
oficialmente sus políticas de TIC en la educación7 y hay una institución
a cargo de la realización de las políticas. Ahora bien, las iniciativas
de los países se clasifican en cuatro áreas: infraestructura,
capacitación, recursos educativos digitales, currículum y evaluación8.
El Gráfico 1 muestra el porcentaje de países divididos según tipos de
acciones. En cuanto a infraestructura, la mayoría de los países realiza
instalaciones de computadoras en las escuelas, promoción de apoyo
técnico y conexión a internet. Asimismo, la entrega de computadoras a
profesores es considera solo por el 42\% de los países mientras que la
entrega de computadoras a alumnos solo es realiza por un tercio de los
países. Respecto a la capacitación, se encuentra que la capacitación de
profesores tanto en el uso como en el uso pedagógico se acciones
prioritarias de la mayoría de los países. Además, el 58\% de los países
realizan la acción de desarrollo de comunidades virtuales de desarrollo
profesionales de profesores. Mientras que la mitad de los países ejecuta
acciones de interacción de TIC en la formación inicial de profesores y
capacitación de profesores en el uso de TIC. Por último, capacitar a los
alumnos en el uso de las TIC es una acción considerada solo por un
tercio de los países. En relación a recursos, la entrega de recursos y
materiales a través del portal educativo es llevado a cabo por la
mayoría de los países; mientras que la mitad de los países desarrolla
contenidos digitales y hace entregas directas a las escuelas de estos
contenidos digitales. Finalmente, respecto a currículum y evaluación, ni
la mitad de los países realizan iniciativas de desarrollo de proyectos
escolares colaborativos, entrega de modelos de uso curricular de TIC o
evaluación de resultados o impacto.

Gráfico 1: Porcentaje de países en los que se implementa según los tipos
de acción

Fuente: Hinostroza en Sunkel \& Trucco (2010).

4.2.1. ANTECEDENTES DE LA POLITICA TICS EN EL PERÚ

\begin{verbatim}
En el Perú, en el gobierno de Alberto Fujimori se inicio una serie de reformas en las políticas TIC en el Perú, seguida de las políticas de los gobiernos de Toledo y el segundo gobierno de García. Estas iniciativas prosperaron hasta el 2012. Sin embargo, esta evolución se vio obstaculizada por la falta de coherencia en las políticas educativas a lo largo del tiempo, lo que resultó en una falta de continuidad entre las políticas iniciadas en 1996 y las adoptadas por el siguiente gobierno. 

   En 1996, el Ministerio de Educación inició dos proyectos de tecnología educativa en escuelas públicas. El programa EDURED de la Red de Educación Unidad 13, que tiene alrededor de 200 escuelas de la ciudad conectadas en redes dial-up con altos costos de acceso. El proyecto INFOESCUELA, un proyecto de robot escolar, es parte del Programa de Mejoramiento de la Calidad de la Educación Primaria (MECEP). Este último cubre 400 escuelas públicas en 17 ciudades del país, y algunas evaluaciones han encontrado que el programa ha tenido un impacto significativo en el aprendizaje9. 
   A finales del gobierno de Fujimori, se crea el Programa de Educación a Distancia (EDIST), que tuvo como objetivo la mejora de la cobertura de la de educación básica en zonas rurales. Luego este programa se instauro el proyecto Huascarán que abarco el EDIST, en el gobierno de Alejandro Toledo. Sin embargo, se identificaron los siguientes problemas: poca claridad en los objetivos educativos e inadecuada planificación, evaluación e implementación de los programas.
\end{verbatim}

4.2.2. AVANCES Y LIMITACIONES DE LA DIGITALIZACIÓN EN EDUCACIÓN EN LA
PANDEMIA Las tecnologías digitales han sido fundamentales para la
continuidad de actividades en medio de la pandemia por COVID-19. En este
sentido, los países de la región idearon métodos para realizar las
actividades escolares a distancia. Sin embargo, esta solución solo es
pensada para quienes tienen acceso a dispositivos y cuentan con
conexión. La división de Educación realizó el Proyecto Sistemas de
Información y Gestión Educativa (SIGED) en la cual examina la
realización de procesos cotidianos de gestión y conocer el nivel de
automatización y aprovechamiento digital. Existen cinco condiciones
digitales: conectividad en las escuelas, plataformas digitales, tutoría
virtual, paquetes de recursos digitales y repositorio central de
contenido digital10. Rieble-Aubourg \& Viteri (2020) encuentran que solo
Uruguay cumple con las cinco condiciones digitales de SIGED; Chile,
Colombia y Argentina cumplen con dos de estas (paquetes de recursos
digitales y repositorio de contenido digital); mientras que Perú y
México solo cumplen con la condición de repositorio de contenido
digital. Según la CEPAL (2020), en la región, el 46\% de los niños y
niñas entre las edades de 5 y 12 años no tiene conexión a Internet. En
otras palabras, 32 millones de niños y niñas son excluidos de las clases
online a distancia. Asimismo, Paraguay, El Salvador, Bolivia y Perú son
los países en donde el 90\% de niños y niñas de los hogares más pobres
no tienen conectividad a internet. Mientras que Chile, Argentina, Costa
Rica y Brasil presentan mejores indicadores de conectividad en niños y
niñas. En los estudiantes de áreas rurales, según los datos PISA 2018,
en Chile (86\%), Uruguay (82\%) y Brasil (73\%) presentan mejores
indicadores en cuanto al acceso a internet. Y los casos menos favorables
son Perú (36\%) y Colombia (35\%).\\
En cuanto a acceso de dispositivos digitales en los hogares, la CEPAL
(2020, a) encuentra que existe desigualdades entre los niveles
socioeconómicos. Por un lado, alrededor del 80\% los estudiantes de
niveles socioeconómicos más altos poseen computadoras. Por otro lado,
solo entre 10\% y 20\% de los estudiantes de los niveles socioeconómicos
más bajos tienen estos dispositivos. En cuanto al acceso del computador
del hogar para realizas las tareas de la escuela, se encuentra que en
Chile (61\%) y Uruguay (55\%) el acceso en los grupos más vulnerables11
es menos limitado en comparación con en el acceso en los grupos
vulnerables de México (13\%) y Perú (7\%)12. Según los datos de PISA
2018, los escolares de 15 años realizaban una serie de actividades con
las TIC tales como uso de aplicaciones de aprendizaje o sitios web,
realizar deberes escolares en una computadora, descargar o subir
material en el sitio web de la institución, usar las redes sociales para
comunicarse, etc. En este análisis de las actividades se encuentran que,
en todas estas a mayor nivel socioeconómico, mayor proporción de
estudiantes que realizan las actividades con las TIC (CEPAL, 2020a);
dicho de otra manera, los escolaress de 15 años de mayor nivel
socioeconómico ya tienen experiencia previa antes del confinamiento de
la pandemia y para estos la adaptación al nuevo sistema de educación
será menos costosa. Según Trucco y Palma (2020) en la adolescencia
empieza la aproximación a Internet en actividades relacionadas a
entretenimiento; de modo que la continuación de la enseñanza virtual a
través de internet es más complicada para los niños y niñas de primaria.
4.3 SITUACIÓN PERUANA En el Gráfico 2, se puede apreciar que ha habido
un aumento de presupuesto asignado a educación desde el 2010. Sin
embargo, el presupuesto asignado a educación como porcentaje del PBI en
el Perú se encuentra por debajo de los países latinoamericanos
(Ñopo,2018). Asimismo, se encuentra que el gasto en educación en el año
2017 y 2018 fue de 3.94\% y 3.72\% del PBI respectivamente. Es decir,
hay una reducción de gasto en educación.

Gráfico 2: Presupuesto Institucional Modificado (PIM) asignado a
educación como porcentaje del PBI

Fuente: Datosmacro.com. Elaboración propia. FALTA ACTUALIZAR ESTE GRAF
(REV 15/09)

\begin{verbatim}
  Por otro lado, el Ministerio de Desarrollo e Inclusión Social (MIDIS) mediante el programa Qali Warma proporcionan alimentos saludables para 4 millones de educación básica regular13. De esta forma, las instituciones educativas estatales no solo son espacio de aprendizaje, sino que velan por el bienestar y desarrollo de los estudiantes. 
   En cuanto a datos del 2019, en la Tabla 6 se muestran los datos relevantes con respecto a la composición de estudiantes según nivel educativo. [26]El 17.09% de los estudiantes se encuentran en educación inicial; el 47.47% se encuentra en primaria y el 36,44% restante está en secundaria. Asimismo, el 49.1% de la muestra es mujer y el resto es hombre. En cuanto a la distribución según tipo de gestión de la escuela es la siguiente: 87.41% de los estudiantes se estudian en instituciones educativas estatales y 12.59% en instituciones educativas privadas (Ver Tabla 7).
   
\end{verbatim}

Tabla 6: Nivel de educación

Nivel EducativoFreq.PorcentajeEducación
inicial2,27917.09Primaria6,19846.47Secundaria4,86036.44Total13,337100Fuente:
ENAHO 2019. Elaboración propia.

Tabla 7: Gestión de la escuela

Gestión de la escuelaFreq.PorcentajeEstatal11,65387.41No
estatal1,67812.59Total13,331100Fuente: ENAHO 2019. Elaboración propia.

\begin{verbatim}
   En cuanto al área geográfica, se encuentra que el 44.51% de los estudiantes son del área rural y el 55.49% pertenecen al área urbana. Asimismo, en el área rural, el 98.40% de estudiantes se encuentra estudiando en escuelas públicas y el 1.60% se encuentra estudiando en escuelas privadas; mientras que, en el área urbana, el 78.6% de los estudiantes asiste a escuelas estatales y el 21.4% lo hace a escuelas privadas (Ver Tabla 8). Entonces, en el área rural, el principal soporte viene del sector público en su mayoría.}
\end{verbatim}

Tabla 8: Ámbito geográfico - gestión de la escuela

~Gestión de la Escuela~Ámbito geográficoEstatal No
estatalTotalRural43.80.7144.51Urbano43.6111.8855.49Total87.4112.59100Fuente:
ENAHO 2019. Elaboración propia.

\begin{verbatim}
   En cuanto al acceso de Internet se encuentra que, en el área rural, el 23.16% de los estudiantes tiene acceso al Internet. Mientras que, en el área urbana, el 76.84% de los estudiantes accede al Internet.  Asimismo, se encuentra que solo el 0.03% de los estudiantes del área rural acceden a Internet desde sus casas; mientras que, en el área urbana, el 32,22% de los estudiantes acceden a internet en sus hogares. En relación a uso de Internet, el 23% de los estudiantes de área rural hace uso de este servicio y en el caso del área urbana, el 63% utiliza este servicio (Ver Tabla 9).
\end{verbatim}

Tabla 9: Ámbito geográfico - Acceso de Internet y Uso de Internet
~Acceso a Internet~Uso de Internet~Ámbito
geográficoOtroHogarTotalSiNoTotalRural22.40.7623.1610.4834.4244.91Urbano52.0724.7776.8434.7720.3355.09Total74.4725.5310045.2554.75100Fuente:
ENAHO 2019. Elaboración propia.

\begin{verbatim}
En el Perú, a raíz del distanciamiento social por el Covid-19 la educación se impartió a distancia, por medio de plataformas o por el programa Aprendo en Casa para escuelas públicas de educación básica regular. Según los datos de la ENAHO del segundo trimestre, el 55,3% de los estudiantes entre 3 a 16 años recibieron las clases por televisión, el 19,2% lo hace por radio y el 31,5% lo hace mediante la plataforma virtual. En las zonas urbanas, el uso del medio televisivo para acceder al contenido de las clases asciende a 60,7%. La radio es utilizada por el 9,9%; la plataforma virtual es usado por el 38,3% y el 16,4% lo hace mediante el WhatsApp. Mientras que, en el área rural, el 42% de los estudiantes acceden a sus clases por radio; el 40,3% hace uso de la televisión y el 30% lo hace mediante WhatsApp14.
   En el 2019, los individuos entre las edades de 6 a 17 años, el 30,5% de los escolares accedía a internet en sus hogares; el 11,2% lo hacía en su establecimiento educativo y el 25,8% lo hacía mediante una cabina pública. Asimismo, el 75,8% de los estudiantes de 12 a 17 años tienen acceso del internet; por otro lado, el 53,8% de los estudiantes de 6 a 11 años de edad tienen acceso a internet (INEI, 20202b[27]).
\end{verbatim}

\begin{enumerate}
\def\labelenumi{\arabic{enumi}.}
\setcounter{enumi}{4}
\item
  HIPÓTESIS{[}28{]} El interés de la investigación es conocer las
  brechas digitales que los estudiantes atraviesan durante la pandemia
  desde dos dimensiones: (1) el uso a Internet; y (2) el uso específico
  que se da al Internet. Entonces, se plantean cuatro hipótesis de
  trabajo que han guiado la investigación: H1. Las estudiantes mujeres
  tienen menor acceso a las TIC. H2. La posesión de recursos
  tecnológicos e internet por si sola, no permite mejoras en el
  aprendizaje de {[}29{]}los estudiantes H3. Los estudiantes con un
  nivel socioeconómico alto tendrán mejores niveles de acceso y uso a
  las TIC. H4. Los estudiantes con pocos miembros en el hogar, tienen
  mayor acceso y uso a las TIC. H5. Los estudiantes de secundaria tienen
  mejores habilidades digitales que los estudiantes de primaria e
  inicial.
\item
  LINEAMIENTO METODOLÓGICOS En esta sección se presentan los datos.
  Luego se presenta el modelo empírico a ser usado para estimar los
  determinantes de acceso, uso y apropiación de las TIC por parte de los
  estudiantes durante el 2020. 6.1. MUESTRA En la presente
  investigación, se emplea los datos extraídos de los módulos uno
  (Características de la Vivienda y del Hogar), dos (Características de
  los Miembros del Hogar) y tres (Educación) de la Encuesta Nacional de
  Hogares ENAHO (2020). La población a estudiar está conformada por
  14125 estudiantes del Perú, entre las edades de 6 a 16 años de
  edad{[}30{]}.
\end{enumerate}

6.2. METODOLOGÍA Y MODELO EMPÍRICO{[}31{]} A fin de analizar si los
estudiantes cuentan con los recursos tecnológicos suficientes para
continuar sus estudios, se realizan estimaciones probit. La variable
dependiente es una variable categórica que mide la conexión a Internet
(DNET= 1, si accede al servicio y DNET = 0, caso contrario. Asimismo,
las variables dependientes serán la edad, sexo, idioma, departamento,
ámbito geográfico (revisar anexo1). Mientras que para los otros dos
subobjetivos{[}32{]} se toma como variable dependiente a una variable
categórica que mide el uso de Internet.{[}33{]} Las variables
dependientes serán la edad, sexo, idioma, departamento, ámbito
geográfico (revisar Anexo1). Fase Inicial Para el primer objetivo
inicial se plantea la siguiente metodología. Siguiendo a Tello (2018),
en la cual primero muestra el acceso a la infraestructura con la cual
conectarse a Internet{[}34{]} (DTIC) . ?DTIC?\_h=X\_h
?\_h+?\_h;(h=1,2,3) ?DTIC?\_h\textgreater? si DTIC=1, de otra manera
?DTIC?\_h=0 Donde: ? h=1, si el individuo posee computador/laptop (DCPD)
? h=2,si el individuo posee de celular (DCEL) ? h=3,si el individuo
posee de internet (DINT) DCPD=?X'?\_h ??'?\_h+??'?\_h;h=1, DCPD=1, de
otra manera DCMD=0 ?DINT?\_h=?X'\,'?\_h ??'\,'?\_h+??'\,'?\_h, si DCMP=1
y DNET=1, de otra manera DINT=0 Tomando en cuenta, nuestros dos últimos
objetivos específicos, se sigue la metodología de Lera-López et
al.~(2011) se realizará un análisis en dos fases. En la primera se mide
los determinantes de uso de Internet por los estudiantes mediante un
modelo probit; mientras que, en la segunda fase, se estudia como los
determinantes afectan la frecuencia de uso, en esta parte se requiere
utilizar el modelo de corrección de sesgo de Heckman (1979). Primera
fase U\_ij=X\_ij\^{}' ?\_ij+?\_ij; j=0,1; i=1,2,….,n (1) Se denotará
Y\_i=1, si el individuo usa internet; mientras que Y\_i=0 es el evento
contrario, es decir, el individuo i no usa internet dado que tiene
computador. Reescribiendo (1) se tiene la siguiente ecuación:
Y\_ij=X\_ij\^{}' ?\_ij+?\_ij; i=1,2,….,n (2) Donde X\_i\^{}' representa
el vector de variables explicatorias del individuo i, el termino ?\_i
son los errores estándares y ?\_i es el vector de coeficientes, el cual
será hallado de la siguiente manera. Dado los errores estándares
normales, se tiene que: Prob(Y\_i=1)=Prob(U\_i1\textgreater?Ui?\_0
)=Prob(X\_i1\^{}' ?\_1+?\_i1\textgreater X\_i0\^{}' ?\_0+?\_i0 )
=Prob(X\_i\^{}' (?\_1-?\_0 )+(?\emph{i1-?\emph{i0 )\textgreater0)
=Prob(X\_i\^{}'
?+?\emph{i\textgreater0)=Prob(?\emph{i\textgreater?-X?\emph{i\^{}'
?)=F(X\_i\^{}' ?) O También, Prob(Y\_i=1)= 1-F(-X\_i\^{}' ?) Y,
Prob(Y\_i=0)= 1-F(X\_i\^{}' ?) Donde F(.) es una función de distribución
acumulada para los términos de error ?\emph{i. Cada Y\_ij observada es
modelada usando un modelo probit y que es asociado con la función de
máxima verosimilitud que puede ser expresada de la siguiente manera:
L=?}(Y}(i=1)) F(X\_i\^{}' ?)?}(Y}(i=0)) {[} 1-F(X\_i\^{}' ?){]} Tomando
logaritmos: logL=?}(Yi=1) log{[}(exp(X\_i\^{}' ?))/(1+exp(X\_i\^{}'
B)){]}+?}(Yi=0) log{[}1/(1+exp(X\_i\^{}' B)){]} Resolviendo esta
ecuación se obtienen ?~?. Segunda fase Se tiene el siguiente modelo
probit ordenado: D\_i=Z\_i'a+u\_i Se tiene que si Y\_i=1 satisface el
modelo; es decir, cuando el estudiante usa internet. Además, Z\_i es el
vector de variables explicativas para el individuo i y a está asociado a
al vector de coeficientes. Se asume que u\_i son el error estándar. Se
utiliza el modelo de Heckman (1979) para corregir el error de sesgo.
Para ello, se introduce el parámetro ?\_i, que representa el ratio de
Mills hallado en la primera fase. ?\_i=(F(X\_i\^{}' ?))/(?(X\_i\^{}' ?))
En resumen, se va a utilizar un modelo de dos etapas. En la primera
etapa se utiliza un modelo probit binomial, en esta parte se identifica
los factores que permiten el uso de Internet Y\_i en los estudiantes.
Mientras que, en la segunda etapa, el modelo busca determinar los
factores que explica la frecuencia de uso de internet (D\_i). 6.3
DEFINICIÓN DE VARIABLES

\begin{enumerate}
\def\labelenumi{\arabic{enumi}.}
\setcounter{enumi}{6}
\item
  CONCLUSIONES
\item
  BIBLIOGRAFÍA Álvarez–Sigüenza, J. (2019). Nativos digitales y brecha
  digital: Una visión comparativa en el uso de las TIC. Revista de la
  Asociación Española de Investigación de la Comunicación, 6(1),
  203-223.
\end{enumerate}

Banco Mundial (1999). Entering the 21st century: world development
report 1999/2000. Washington, D.C.: Oxford University Press.

Barrantes, R., \& Vargas, E. (2017). ¿Caminos distintos y destinos
iguales?: Análisis de la convergencia en patrones de uso de Internet
entre diferentes grupos etarios.

Bernabeu, N. (1997). Educar en una sociedad de
información.~Comunicar,~4(8), 73-82.

Brynin, M. (2006). The neutered computer. En R. Kraut, M. Brynin y S.
Kiesler (Eds.), Computers, phones, and the Internet (pp.~84-96). Oxford
University Press.

Calderón, D. C. (2019a). Una aproximación a la evolución de la brecha
digital entre la población joven en España (2006-2015). Revista Española
de Sociología, 28(1), 27-44.

Castaño, J. Duart J. \& Sancho T. (2012) Una segunda brecha digital
entre el alumnado universitario, Cultura y Educación, 24:3, 363-377

Castells, M. (2000). La era de la información: economía, sociedad y
cultura, Vol. I La sociedad red. 2a ed., Alianza Editorial, S. A,
España.

Castells, M. (2001). La galaxia Internet. Plaza y Janés, España.

Castell, M., Tubella, I., Sancho, T. \& Roca, M. (2007). La transició a
la societat xarxa. Barcelona: Ariel

Chen, W., \& Wellman, B. (2004). The global digital divide–within and
between countries.~IT \& society,~1(7), 39-45.

CEPAL, N. (2008). Desarrollo e igualdad: el pensamiento de la CEPAL en
su séptimo decenio. Textos seleccionados del período 2008-2018.

CEPAL, N. (2020). La educación en tiempos de la pandemia de COVID-19.

CEPAL, N. (2020b). Universalizar el acceso a las tecnologías digitales
para enfrentar los efectos del COVID-19.

Clafin, B. (13 de octubre del 2000) ``El ABC y D de la brecha digital'',
Bruce Claflin, Diario Reforma, Sección Negocios. Recuperado el 5 de
octubre del 2020 de
https://reforma.vlex.com.mx/vid/bruce-abc-brecha-digital-81068877

Claro, M., Espejo, A., Jara, I. y Trucco, D. (2011). Aporte del sistema
educativo a la reducción de las brechas digitales. Una mirada desde las
mediciones PISA. Chile: CEPAL.

Crovi-Drueta, D. (2008). Dimensión social del acceso, uso y apropiación
de las TIC.~Contratexto, (016), 65-79.

Cueto, S., León J. \& Felipe C. (2020) Acceso a dispositivos y
habilidades digitales de dos cohortes en el Perú. Análisis y Propuestas.

Cuevas, F \& Alvarez, V. (2009). Brecha digital en la educación
secundaria: el caso de los estudiantes costarricenses. Universidad de
Costa Rica. Costa Rica.

Datosmacro.com (s.f) Perú - Gasto público Educación. Recuperado el 5 de
octubre del 2020 de:
https://datosmacro.expansion.com/estado/gasto/educacion/peru

Datum Internacional (s. f). Internet en el Perú. Recuperado el 28 de
agosto del 2020 de:
http://www.datum.com.pe/new\_web\_files/files/pdf/Internet.pdf DiMaggio,
Paul et al.~(2004), ``From unequal access to differentiated use: A
literature review and agenda for research on digital inequality'', en
Neckerman, Kathryn {[}ed.{]}, Social inequality, Estados Unidos: Russell
Sage Foundation.

Duplaga, M. (2017). Digital divide among people with disabilities:
Analysis of data from a nationwide study for determinants of Internet
use and activities performed online.~PloS one,~12(6), e0179825.

Fernández Enguita, M. (30 de marzo del 20202. ``Una pandemia
imprevisible ha traído la brecha previsible''. Cuaderno de Campo.

Garduño, R. (2004). La sociedad de la información en México frente al
uso de Internet. Revista Digital Universitaria, 5(8), 1-13.

Gehring, J. (2001). Not enough girls. Education Week, 20(35), 18-19.

Gestión (22 de setiembre del 2020) ``Unos 300,000 escolares peruanos
desertan en medio de la pandemia''. Recuperado el 18 de octubre del 2020
de
https://gestion.pe/peru/unos-300000-escolares-peruanos-desertan-en-medio-de-la-pandemia-noticia/

Grazzi M., S. Vergara (2009). ICT Access in Latin America: Evidence from
Household Level. Project Observatory for the Information Society in
Latin American and the Caribbean (OSILAC), Third Phase. Chile.
Recuperado el 4 de enero del 2021 de:
https://mpra.ub.uni-muenchen.de/33266/

Grisales, N. (2011). La brecha cognitiva: una realidad educativa que va
más allá de la brecha digital entre las instituciones urbanas y rurales
de manizales. Revista Latinoamericana de Estudios Educativos (Colombia),
7(2),37-56. Recuperado el 25 de noviembre del 2020 de:
https://www.redalyc.org/articulo.oa?id=1341/134125454004

Grubesic, T. H. (2008). The spatial distribution of broadband providers
in the United States: 1999–2004.~Telecommunications Policy,~32(3-4),
212-233. Recuperado el 4 de enero del 2021:
https://www.sciencedirect.com/science/article/pii/S0308596108000086

Haugland, S. W. (2000). Early childhood classrooms in the 21st century:
Using computers to maximize learning. Young Children, 55(1), 12-18.

Heckman, J. (1979). Sample selection bias as a specification error.
Econometrica 47: 153–61.

Hohlfeld, T., Ritzhaupt, A., and Barron, A. (2008). Examining the
digital divide in K-12 public schools: Four-year trends for sup- porting
ICT literacy in Florida. Computers and Education. 51(4): 1648-1663.

Instituto Nacional de Estadística e Informática. Matrícula escolar en el
sistema educativo, según nivel, modalidad y sector, 2008 – 2018, Lima,
Perú.

James, J. (2001). Bridging the digital divide with low-cost information
technologies.~Journal of Information Science,~27(4), 211-217.

Jiménez, R., Vega, L. y Vico, A (2016). Habilidades en internet de
mujeres estudiantes y su relación con la inclusión digital: Nuevas
brechas digitales. Education in the Knowledge Society, 17(3),29-48.
Recuperado el 30 de noviembre del 2020 de:
https://doi.org/10.14201/eks20161732948

Kelley-Salinas, G. (2000). Different educational inequalities: ICT an
option to close the gaps.~Learning to bridge the digital divide, 21-36.

Kiiski, S., \& Pohjola, M. (2002). Cross-country diffusion of the
Internet.~Information Economics and Policy,~14(2), 297-310. Recuperado
el 4 de enero del 2021 de:
https://www.sciencedirect.com/science/article/abs/pii/S0167624501000713

Kuttan, A. P., \& Peters, L. (2003).~From digital divide to digital
opportunity. Scarecrow Press, Inc.

Lera-López, F., Billon, M., \& Gil, M. (2011). Determinants of Internet
use in Spain.~Economics of Innovation and New Technology,~20(2),
127-152.

Lera-López, F., Hernández Nanclares, N., \& Blanco Vaca, C. (2003). La
``brecha digital'': un reto para el desarrollo de la sociedad del
conocimiento.~Revista de Economía Mundial, 2003, (8). Págs. 119-142.

Losh, S. C. (2003). Gender and educational digital chasms in computer
and internet access and use over time: 1983–2000.~IT \& Society,~1(4),
73-86. Recuperado el 3 de enero del 2021 de:
http://citeseerx.ist.psu.edu/viewdoc/download?doi=10.1.1.385.9971\&rep=rep1\&type=pdf

Madden, G., \& M. Simpson. 1997. Residential broadband subscription
demand: An econometric analysis of Australian choice experiment data.
Applied Economics 29: 1073–8.

Márquez, A. M., Acevedo J. A. A., \& Castro, D. C. (2016). Brecha
digital y desigualdad social en México.~Economía Coyuntural, Revista de
temas de perspectivas y coyuntura,~1(2), 89-136.

McKeown, L., Noce, A., \& Czerny, P. (2007).~Factors associated with
Internet use: Does rurality matter? Statistics Canada, Agriculture
Division.

Mendoza Z, D. M. (2017). Análisis del acceso a Internet de los
estudiantes de Bachillerato en Ecuador.

MIDIS (17 de setiembre del 2019) Midis Qali Warma y Minedu garantizan
una alimentación saludable todos los días del año escolar. Recuperado el
8 de diciembre del 2020 de:
https://www.qaliwarma.gob.pe/noticias/midis-qali-warma-minedu-garantizan-una-alimentacion-saludable-todos-los-dias-del-ano-escolar/

Ministerio de Educación (s. f.) APUESTAS DEL SECTOR EDUCATIVO 2020
Recuperado el 15 de junio del 2021 de:
http://escale.minedu.gob.pe/c/document\_library/get\_file?uuid=248fc583-1778-4c93-b48c-48684c2733d9\&groupId=10156

Mills, B.F., \& B.E. Whitacre. (2003). Understanding the
non?metropolitan?metropolitan digital divide. Growth and Change 34:
219–43. Monge, R \& Hewitt, J (2004), Tecnologías de la información y las
comunicaciones (TICs) y el futuro desarrollo de Costa Rica: el desafío
de la exclusión, Costa Rica, Academia Centroamericana.

Murthy, V. N., Nath, R., \& Soleimani, M. (2015). Long term factors of
internet diffusion in Sub-Saharan Africa: A panel co-integration
analysis.~Journal of International Technology and Information
Management,~24(4), 5.

Norris, P. (2001).~Digital divide: Civic engagement, information
poverty, and the Internet worldwide. Cambridge university press.

Ñopo, H. (2018). Análisis de la inversión educativa en el Perú desde una
mirada comparada.

OECD (2007), ``Broadband and ICT Access and Use by Households and
Individuals'', OECD Digital Economy Papers, No.~135, OECD Publishing,
Paris. http://dx.doi.org/10.1787/230666254714

Peres, W. y Hilbert, M. (2009). La Sociedad de la Información en América
Latina y el Caribe. Desarrollo de las tecnologías y tecnologías para el
desarrollo. CEPAL, Chile.

Prensky, M. (2011). Enseñar a nativos digitales. Madrid: Ediciones SM.

Quiroz, M. (2014). Las brechas digitales en las aulas
peruanas.~miradas,~1(12).

Rhee, K.Y. \& Kim, W. B. (2004). The adoption and use of the Internet in
South Korea. Journal of Computer-Mediated Communication, 9(4), 169-186.

Riddlesden, D., \& Singleton, A. D. (2014). Broadband speed equity: A
new digital divide? Applied Geography, 52(C), 25–33.

Rieble-Aubourg, S., \& Viteri, A. (2020). COVID-19: ¿Estamos preparados
para el aprendizaje en línea?~Nota CimA, (20).

Rodríguez Gallardo, J. A. (2006). La brecha digital y sus
determinantes{[}35{]}{[}36{]}. UNAM.

Rodríguez G., C., \& Sandoval M., D. (2017). Estratificación digital:
acceso y usos de las TIC en la población escolar chilena.~Revista
electrónica de investigación educativa,~19(1), 20-34.

Sádaba, C. (2010). El perfil del usuario de Internet en
España.~Psychosocial Intervention,~19(1), 41-55.

Schmelkes. S. (31 de marzo del 2020) ``Clases digitales marcarán una
brecha de aprendizaje''. La Jornada. Recuperado el 20 de noviembre del
2020 de
https://www.jornada.com.mx/ultimas/sociedad/2020/03/31/clases-digitales-marcaran-una-brecha-de-aprendizaje-schmelkes-5346.html

Schneir, J. R., \& Xiong, Y. (2016). A cost study of fixed broadband
access networks for rural areas.~Telecommunications Policy,~40(8),
755-773.

Selwyn, N. (2004). Reconsidering political and popular understandings of
the digital divide. New Media \& Society, 6(3), 341–362.

Statista (2018) ¿Cuántos usuarios de Internet hay en América Latina?
Recuperado el 20 de setiembre del 2020 de:
https://es.statista.com/grafico/13903/cuantos-usuarios-de-internet-hay-en-america-latina/

Steinberg, C. (2013). Televisión, Internet y educación básica.~Revista:
Programa Tics y Educación Básica,~1.

Sunkel, G. (2006). Las Tecnologías de la Información y la Comunicación
(TIC) en América Latina: Una exploración de Indicadores. Santiago de
Chile: Serie Políticas Sociales.

Sunkel, G., \& Trucco, D. (2010).~Nuevas tecnologías de la información y
la comunicación para la educación en América Latina: riesgos y
oportunidades. Cepal.

Sunkel, G., Trucco, D. y Möller, S. (2011). Aprender y enseñar con las
tecnologías de la información y las comunicaciones en América Latina:
potenciales beneficios. Chile: CEPAL.

Tello, E. (2007). Las tecnologías de la información y comunicaciones
(TIC) y la brecha digital: su impacto en la sociedad de
México.~International Journal of Educational Technology in Higher
Education (ETHE).

Tello, M. (2018). Brecha digital en el Perú: Diagnóstico, Acceso, Uso e
Impactos. Lima: INEI.

Trucco, D. y A. Palma (2020), ``Infancia y adolescencia en la era
digital: un informe comparativo de los estudios de Kids Online del
Brasil, Chile, Costa Rica y el Uruguay'', Documentos de Proyectos
(LC/TS.2020/18), Santiago, Comisión Económica para América Latina y el
Caribe (CEPAL).

UNESCO (2005) Informe mundial 2005. Hacia las Sociedades del
Conocimiento, París, UNESCO.

UNESCO (2014). Indicadores UNESCO de cultura para el desarrollo: manual
metodológico. Paris: UNESCO.

Unión Internacional de Telecomunicaciones (UIT) (2009). Measuring the
Information Society. The ICT Development Index. UIT, Suiza.

Van Deursen, Alexander \& van Dijk, Jan (2013), ``The digital divide
shifts to differences in usage'', en New media y society, vol.~16, núm.
3, Estados Unidos: Sage.

Van Dijk, J. (2017). Digital divide: impact of access. En P. Rössler,
C.A. Hoffner y L. van Zoonen (eds.), De International Encyclopedia of
Media Effects (pp.~1-11), Chichester, UK: John Wiley

Vicente, M.R., and A.J. López. 2008. Some empirical evidence on Internet
diffusion in the New Member States and Candidate Countries of the
European Union. Applied Economics Letters 15: 1015–18.

?Warschauer, M. (2004).~Technology and social inclusion: Rethinking the
digital divide. MIT press.

World Statistics (s. f.). http://world-statistics.org/index.php

World Resources Institute. Creating digital dividends. 2000. Recuperado
el 28 de noviembre del 2020 de: http://www.digitaldividend.org/.

Zamora S, G., \& Rios E, G. L. (2019) ¿Conectar para incluir?: brecha
digital en las personas con discapacidad. Análisis de su uso y
apropiación de Internet desde un enfoque de capacidades.

Zhao, Ling et al.~(2010), ``Internet inequality: The relationship
between high school students’ Internet use in different locations and
their Internet self-efficacy'', en Computers y Education, vol.~55, Reino
Unido: Elsevier.

\begin{enumerate}
\def\labelenumi{\arabic{enumi}.}
\setcounter{enumi}{8}
\tightlist
\item
  ANEXOS Anexo1: Variables a estimar Anexo1: Variables a estimar
  1.-Tecnologías de la Información y
  ComunicaciónVariableDescripciónDNETVariable binaria (1 si el
  estudiante dispone de Internet)DCELVariable binaria (1 si el
  estudiante dispone de celular)DCMDVariable binaria (1 si el estudiante
  dispone de computadora/laptop)2.-Nivel socioeconómicos y
  educativosEST\_AVariable dummy (1 si el estudiante pertenece en el
  estrato sociodemográfico A; 0 caso contrario).EST\_BVariable dummy (1
  si el estudiante pertenece en el estrato sociodemográfico B; 0 caso
  contrario).EST\_CVariable dummy (1 si el estudiante pertenece en el
  estrato sociodemográfico C; 0 caso contrario).EST\_DVariable dummy (1
  si el estudiante pertenece en el estrato sociodemográfico D; 0 caso
  contrario).EST\_EVariable dummy (1 si el estudiante pertenece en el
  estrato sociodemográfico E; 0 caso contrario).GRADOVariable que indica
  los años de escolaridad del estudiante.ESCUELAVariable binaria (1 si
  el estudiante asiste a una escuela pública; 0 en caso asista a una
  escuela privada).2.- SociodemográficosEDADVariable edad en años
  cumplidos SEXOVariable que denota el sexo del estudiante (1 si es
  mujer, 0 si es hombre).IDIOMAVariable que denota la lengua materna del
  estudiante (1 si habla castellano, 0 otros).MIEMBROSVariable que
  muestra el número de miembros en el hogar.3.- Geográficas e
  infraestructuracostaVariable dummy (1 si el estudiante pertenece a la
  Costa; 0 en caso contrario).sierraVariable dummy (1 si el estudiante
  pertenece a la Sierra; 0 en caso contrario).selvaVariable dummy (1 si
  el estudiante pertenece a la Selva; 0 en caso contrario).D1Variable
  dummy (1 si el estudiante pertenece a la Amazonas; 0 en caso
  contrario).D2Variable dummy (1 si el estudiante pertenece a la Ancash;
  0 en caso contrario).D3Variable dummy (1 si el estudiante pertenece a
  la Apurímac; 0 en caso contrario).D4Variable dummy (1 si el estudiante
  pertenece a la Arequipa; 0 en caso contrario).D5Variable dummy (1 si
  el estudiante pertenece a la Ayacucho; 0 en caso contrario).D6Variable
  dummy (1 si el estudiante pertenece a la Cajamarca; 0 en caso
  contrario).D7Variable dummy (1 si el estudiante pertenece a la Callao;
  0 en caso contrario).D8Variable dummy (1 si el estudiante pertenece a
  la Cusco; 0 en caso contrario).D9Variable dummy (1 si el estudiante
  pertenece a la Huancavelica; 0 en caso contrario).D10Variable dummy (1
  si el estudiante pertenece a la Huánuco; 0 en caso
  contrario).D11Variable dummy (1 si el estudiante pertenece a la Ica; 0
  en caso contrario).D12Variable dummy (1 si el estudiante pertenece a
  la Junín; 0 en caso contrario).D13Variable dummy (1 si el estudiante
  pertenece a la La Libertad; 0 en caso contrario).D14Variable dummy (1
  si el estudiante pertenece a la Lambayeque; 0 en caso contrario).
  D15Variable dummy (1 si el estudiante pertenece a la Lima; 0 en caso
  contrario).D16Variable dummy (1 si el estudiante pertenece a la
  Loreto; 0 en caso contrario).D17Variable dummy (1 si el estudiante
  pertenece a la Madre de Dios; 0 en caso contrario).D18Variable dummy
  (1 si el estudiante pertenece a la Moquegua; 0 en caso
  contrario).D19Variable dummy (1 si el estudiante pertenece a la Pasco;
  0 en caso contrario).D20Variable dummy (1 si el estudiante pertenece a
  la Piura; 0 en caso contrario).D21Variable dummy (1 si el estudiante
  pertenece a la Puno; 0 en caso contrario).D22Variable dummy (1 si el
  estudiante pertenece a la San Martín; 0 en caso contrario).D23Variable
  dummy (1 si el estudiante pertenece a la Tacna; 0 en caso
  contrario).D24Variable dummy (1 si el estudiante pertenece a la
  Tumbes; 0 en caso contrario). D25Variable dummy (1 si el estudiante
  pertenece a la Ucayali; 0 en caso contrario).FAMILIAVariable dummy (1
  si la vivienda es multifamiliar; 0 en caso contrario).DLUZVariable
  dummy (1 si la vivienda cuenta con servicios de electricidad; 0 en
  caso contrario).
\end{enumerate}

1 Según los datos del INEI, aproximadamente 2 millones de estudiantes
asisten a colegios privados, mientras que 6 millones asisten a colegios
públicos en todo el territorio peruano. 2 Cfr. World Statistics (s. f.).
http://world-statistics.org/index.php 3 Cfr. Datum Internacional (s. f.)
4 Cfr. Zamora \& Rios (2019) 5 Cfr. Mendoza (2017) 6 El Plan de Acción
Regional sobre la Sociedad de la Información en América Latina y el
Caribe 7 Perú, Paraguay, Nicaragua, México, Guatemala, Bolivia, Uruguay,
Colombia y Chile. 8 Cfr. Sunkel \& Trauco (2010). 9 Informe Técnico
Pedagógico sobre la Aplicación del Proyecto INFOESCUELA en los Centros
Educativos Pilotos 1996 y seleccionados 1997; Estudio del Impacto
Educacional de los Materiales. LEGO Dacta - INFOESCUELA - MED. 10 Cfr.
Rieble-Aubourg \& Viteri (2020) 11 Quintil I. 12 Cfr. Rieble-Aubourg \&
Viteri (2020) 13 Cfr MIDIS (2019). 14 Estos medios no son excluyentes.
Cfr INEI (2020b) Concordancia las/es? Plural o singular? TIC:
tecnologías de información y la comunicación revisar presentación de
objetivos específicos.

Tal como están planteados los objetivos, parece que lo que quieres es
describir la situación en cada dimensión de la brecha digital (acceso,
uso y apropiación). Si el objetivo general es comprender los
determinantes, los objetivos específicos debieran apuntar a la misma
lógica. entre quiénes? Oración incompleta. No es claro qué quieres decir
aquí Qué variables? mencionar las variables que estan en el anexo
Revisar redacción done Ajustar de acuerdo a los avance de Tesis2 Evalúa
empezar esta sección con las definiciones que incluyes más abajo, en el
2do y 3er párrafo A medida que vas definiendo esto, ya debieras ir
pensando en las variables necesarias para capturar cada dimensión y
guardarlo para más adelante {[}13R12{]} Considerando tus objetivos
específicos, en esta sección debieras analizar no solo los determinantes
de acceso, también de uso y apropiación. Y en el camino, pues ir
tratando de distinguir de manera más fina qué dimensión de la brecha
digital se afecta. Ok Para ilustrar por qué los determinantes económicos
pueden ser muy o poco importantes.. debieras dar una idea de lo que
significa en términos de costos acceder a las tic.\\
Además, estamos hablando del acceso de los individuos o de grupos
poblacionales? Lo económico tiene que ver con el ingreso de los
individuos o también con el de las zonas en que habitan.. por qué? Lo
contenido aquí es básicamente una síntesis. Tal vez podrías hacer
referencia a literatura empírica, si la hay e identificar si hay un
vacío o espacio para la investigación que tu tesis busca subsanar Listo
No veo una revisión en línea con lo que conversamos en la última reunion
Ya lo he mejorado ¿? Debieras empezar recordando estas definiciones,
pues en el modelo hablas de acceso y uso. No me parece apropiado
presentarlo en una tabla. Dado que son conceptos que ya viste antes,
podría incluirlo como parte de una introducion concpetual.. recordando
que la brecha digital tiene tres dimensions… El análisis de los datos se
realiza en la metodología, no en los hechos estilizados En los hechos
estilizados estableces el contexto en que se desarrolla el estudio. Se
podría incluir también relaciones tentativas entre las variables
dependientes y las explicativas. Esta sección no corresponde a los
hechos estilizados. Debiera estar al inicio del capítulo Aproximación
Empírica o creo que le llamas aproximación metodológica

En la sección de hechos estilizados te falta un hilo conductor. La ideas
aparecen sin haber aparentemente criterio de organización.

Tal vez podrías empezar estableciendo el contexto.. Situación del país
en infraestructura para uso de medios digitales, acceso diferenciado de
la población, Políticas permanente y de coyuntura que influyen en
mejorar la infraestructura y el acceso.. Presencia y perfil de los
estudiantes… tu público objetivo.. Acceso de éstos a infraestructura
digital… uso, apropiación.. Situación en pandemia. Esto y en general la
descipción del grupo objetivo debiera ser descrito antes. Se tiene
estadísticas de la brecha digital por género. Debieras tratar de incluir
información sobre las variables que has identificado como posibles
determinantes de la brecha digital, en especial de aquellas que incluyes
en tus hipótesis Vincular hipótesis con objetivos específicos.

Qué variables dependientes y explicativas te genera cada una de estas
hipótesis? Si tuvieras que poner cada H en forma estructural cómo sería?
Aquí la dependiente es mejora en el aprendizaje? Antes de describir la
información, empieza con el planteamiento metodológico

En este punto, daría más claridad el recordar tus objetivos específicos
y vincular a tu metodología, que debiera ayudarte a lograrlos. Esto es a
lo que llamaste objetvios específicos? Recuerda al lector cuales son.
redacción Revisa redacción.. Qué son DTIC, DCEL, etc? Defínelas.
Completar referenci Listo ---------------

\begin{center}\rule{0.5\linewidth}{0.5pt}\end{center}

\begin{center}\rule{0.5\linewidth}{0.5pt}\end{center}

\begin{center}\rule{0.5\linewidth}{0.5pt}\end{center}

2
